\thispagestyle {empty}
\begin{center}
\begin{minipage}[c][5cm][t]{\textwidth}
\begin{center}
\includegraphics[width=5cm]{../IST_A_RGB_POS.png}
\end{center}

\end{minipage}
\begin{minipage}[t][10cm][c]{\textwidth}
\centering
{\FontMb Laboratório de Introdução à Física Experimental} \\
\paragraph{}
\centering
{\FontLb\Huge \title{Experiência de Thomson}}
\paragraph{}
{\FontMb Determinação experimental da relação $q/m$ do eletrão} \\
\paragraph{}
{\FontMb 2023}
\end{minipage}

\begin{minipage}[c][1.5cm][c]{\textwidth}
\centering
{\FontLn }
\end{minipage}

\begin{minipage}[c][1.5cm][c]{\textwidth}
\centering



\end{minipage}
\begin{minipage}[c][3cm][c]{\textwidth}
\centering
\renewcommand{\arraystretch}{1.4}

\maketitle

\vspace{-5mm}
\hline
\vspace{-3mm}
\begin{center}
\centering
\section*{\centering Objetivos}
    \vspace{-3mm}
\small
\justify
Pretende-se com este trabalho determinar a relação entre a carga e a massa ($q/m$) do electrão. Para esse fim, vamos
estudar a deflexão de um feixe de raios catódicos sob o efeito de um campo eléctrico e de um campo magnético.

    
\end{center}
\hline


\end{minipage}
\begin{minipage}[c][2cm][c]{\textwidth}
\centering

\end{minipage}

\end{center}
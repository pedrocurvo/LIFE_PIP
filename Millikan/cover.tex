\thispagestyle {empty}
\begin{center}
\begin{minipage}[c][5cm][t]{\textwidth}
\begin{center}
\includegraphics[width=5cm]{../IST_A_RGB_POS.png}
\end{center}

\end{minipage}
\begin{minipage}[t][10cm][c]{\textwidth}
\centering
{\FontMb Laboratório de Introdução à Física Experimental} \\
\paragraph{}
\centering
{\FontLb\Huge \title{Experiência de Milikan}}
\paragraph{}
{\FontMb Estimativa da carga elétrica de gotículas de óleo eletrizadas em suspensão num fluido} \\
\paragraph{}
{\FontMb 2023}
\end{minipage}

\begin{minipage}[c][1.5cm][c]{\textwidth}
\centering
{\FontLn }
\end{minipage}

\begin{minipage}[c][1.5cm][c]{\textwidth}
\centering



\end{minipage}
\begin{minipage}[c][3cm][c]{\textwidth}
\centering
\renewcommand{\arraystretch}{1.4}

\maketitle

\vspace{-5mm}
\hline
\vspace{-3mm}
\begin{center}
\centering
\section*{\centering Objetivos}
    \vspace{-3mm}
\small
\justify
Pretende-se com este trabalho determinar a carga eléctrica de pequenas gotas de óleo, tendo como objetivo final mostrar que a carga eléctrica não aparece com uma quantidade qualquer mas sempre como um múltiplo de uma unidade fundamental: a carga do electrão. Deste modo, um corpo electrizado apresenta um excesso de carga de sinal positivo ou negativo, mas cuja valor é sempre um múltiplo do valor da carga elementar $q_{ele}= 1,602176634\cdot 10^{-19}\,$ C.
Traduz-se este facto dizendo-se que a carga eléctrica é \emph{quantizada}.

Dentro das várias experiências elaboradas para mostrar este facto, uma montagem clássica é a do físico americano Robert A. Millikan\footnote{Millikan recebeu o prémio Nobel da Física em 1923 pelos seus trabalhos sobre a determinação da carga do electrão e efeito fotoeléctrico.} (1869-1953), também chamada experiência da gota de óleo.
    
\end{center}
\hline


\end{minipage}
\begin{minipage}[c][2cm][c]{\textwidth}
\centering

\end{minipage}

\end{center}
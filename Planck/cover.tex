\thispagestyle {empty}
\begin{center}
\begin{minipage}[c][5cm][t]{\textwidth}
\begin{center}
\includegraphics[width=5cm]{../IST_A_RGB_POS.png}
\end{center}

\end{minipage}
\begin{minipage}[t][10cm][c]{\textwidth}
\centering
{\FontMb Laboratório de Introdução à Física Experimental} \\
\paragraph{}
\centering
{\FontLb\Huge \title{Espetroscopia e Efeito Fotoelétrico}}
\paragraph{}
{\FontMb Riscas Espetrais e Medição da Constante de Planck} \\
\paragraph{}
{\FontMb 2023}
\end{minipage}

\begin{minipage}[c][1.5cm][c]{\textwidth}
\centering
{\FontLn }
\end{minipage}

\begin{minipage}[c][1.5cm][c]{\textwidth}
\centering



\end{minipage}
\begin{minipage}[c][3cm][c]{\textwidth}
\centering
\renewcommand{\arraystretch}{1.4}

\maketitle

\vspace{-5mm}
\hline
\vspace{-3mm}
\begin{center}
\centering
\section*{\centering Objetivos}
    \vspace{-3mm}
\small
\justify
Pretende-se com este trabalho investigar e fazer uso de várias propriedades da óptica ondulatória, nomeadamente da separação angular das riscas de emissão de lâmpadas espectrais. Utilizando um goniómetro, iremos proceder à medição dos ângulos de refracção de um prisma e de difracção de uma rede, em função do comprimento de onda. A separação das riscas espectrais será também usada para verificar o efeito fotoeléctrico e obter uma medição da constante de Planck.

Como objetivo associado, pretende-se tomar conhecimento e aprender a manusear e a realizar medidas correctamente  com um instrumento óptico de precisão, o \emph{goniómetro}. Este instrumento permite medir ângulos de desvio, por reflexão ou refracção de feixes de raios paralelos, com uma resolução inferior a um minuto de grau.
    
\end{center}
\hline


\end{minipage}
\begin{minipage}[c][2cm][c]{\textwidth}
\centering

\end{minipage}

\end{center}
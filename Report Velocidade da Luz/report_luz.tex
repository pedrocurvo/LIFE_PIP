% ########################################################################################################################################
%
% This is the main latex file. Here we call for inputs from other files. We also define some of the main characteristics of the document.
%
% ########################################################################################################################################
%
% You likely only need to modify the "Main_Content__Write_your_essay_here.tex" file.
%
% ########################################################################################################################################

\documentclass[12pt,a4paper,oneside]{paper}
% Encoding and Language
\usepackage[utf8x]{inputenc}
\usepackage{csquotes}
\usepackage[main=english,portuguese]{babel}
\usepackage{iflang}
\usepackage{ragged2e}
\usepackage{makeidx}
\usepackage{multicol}
\usepackage[export]{adjustbox}
\usepackage{tikz-dependency}
\usepackage{fancyhdr}

% Font Configurations
\renewcommand{\rmdefault}{phv}
\renewcommand{\sfdefault}{phv}
\def\FontLn{% 16 pt normal
  \usefont{T1}{phv}{m}{n}\fontsize{16pt}{16pt}\selectfont}
\def\FontLb{% 16 pt bold
  \usefont{T1}{phv}{b}{n}\fontsize{16pt}{16pt}\selectfont}
\def\FontMn{% 14 pt normal
  \usefont{T1}{phv}{m}{n}\fontsize{14pt}{14pt}\selectfont}
\def\FontMb{% 14 pt bold
  \usefont{T1}{phv}{b}{n}\fontsize{14pt}{14pt}\selectfont}
\def\FontSn{% 12 pt normal
  \usefont{T1}{phv}{m}{n}\fontsize{12pt}{12pt}\selectfont}

% Font Encoding
\usepackage[T1]{fontenc}

% Page Geometry
\usepackage{geometry}	
\geometry{verbose,tmargin=2.cm,bmargin=2.cm,lmargin=2.cm,rmargin=2cm}

% Line Spacing
\usepackage{setspace}
\renewcommand{\baselinestretch}{1.25}

% Graphics and Figures
\usepackage{graphicx}
\usepackage{subfigure}
\usepackage{subfigmat}
\usepackage{float}

% Mathematics and Theorems
\usepackage{amsmath}
\usepackage{amsthm}
\usepackage{amsfonts}
\usepackage{dcolumn}
\usepackage{indentfirst}

% Comments and Verbatim
\usepackage{verbatim}

% Hyperlinks
\usepackage[pdftex]{hyperref}
\hypersetup{
    colorlinks,
    linkcolor=blue,
    anchorcolor=black,
    citecolor=cyan,
    filecolor=black,
    menucolor=black,
    urlcolor=teal,
    bookmarksopen=true,
    bookmarksnumbered=true
}

% Captions and References
\usepackage[figure,table]{hypcap}
\usepackage[format=plain]{caption}
\DeclareCaptionFont{georgia}{\small\fontseries{n}\fontfamily{georgia}\selectfont}
\captionsetup{labelfont=georgia,font=georgia}

% Bibliography
\usepackage[backend=biber,style=apa]{biblatex}

% Acronyms
\usepackage[printonlyused]{acronym}

% Lipsum (for placeholder text)
\usepackage{lipsum}

% Cleveref (for clever references)
\usepackage[\IfLanguageName{english}{english}{portuguese}]{cleveref}

% Colors
\usepackage{xcolor}
\usepackage{color}

% Custom Commands
\newcommand{\gray}[1]{\textcolor{gray}{#1}}

% Equation Numbering
\renewcommand{\theequation}{{\fontseries{n}\fontfamily{georgia}\selectfont\arabic{equation}}}

% Section and Subsection Fonts
\sectionfont{\Large\bfseries\fontfamily{lmss}\selectfont}
\subsectionfont{\large\bfseries\fontfamily{lmss}\selectfont}

\makeindex

\addbibresource{bibliography.bib}
\begin{document}
\pagestyle{plain}

%TC:ignore
% #############################################################################
%
%                           ENTER YOUR NAME, ISTid, AND TITLE
% 
% #############################################################################



\def\title {Velocidade da Luz}


% #############################################################################
%
%               DO NOT MODIFY THE LINES FROM HERE TO THE MAIN DOCUMENT BODY
% 
% #############################################################################

\thispagestyle {empty}
\begin{center}
\begin{minipage}[c][5cm][t]{\textwidth}
\begin{center}
\includegraphics[width=5cm]{../IST_A_RGB_POS.png}
\end{center}

\end{minipage}
\begin{minipage}[t][10cm][c]{\textwidth}
\centering
{\FontMb Laboratório de Introdução à Física Experimental} \\
\paragraph{}
\centering
{\FontLb\Huge \title{Experiência de Milikan}}
\paragraph{}
{\FontMb Estimativa da carga elétrica de gotículas de óleo eletrizadas em suspensão num fluido} \\
\paragraph{}
{\FontMb 2023}
\end{minipage}

\begin{minipage}[c][1.5cm][c]{\textwidth}
\centering
{\FontLn }
\end{minipage}

\begin{minipage}[c][1.5cm][c]{\textwidth}
\centering



\end{minipage}
\begin{minipage}[c][3cm][c]{\textwidth}
\centering
\renewcommand{\arraystretch}{1.4}

\maketitle

\vspace{-5mm}
\hline
\vspace{-3mm}
\begin{center}
\centering
\section*{\centering Objetivos}
    \vspace{-3mm}
\small
\justify
Pretende-se com este trabalho determinar a carga eléctrica de pequenas gotas de óleo, tendo como objetivo final mostrar que a carga eléctrica não aparece com uma quantidade qualquer mas sempre como um múltiplo de uma unidade fundamental: a carga do electrão. Deste modo, um corpo electrizado apresenta um excesso de carga de sinal positivo ou negativo, mas cuja valor é sempre um múltiplo do valor da carga elementar $q_{ele}= 1,602176634\cdot 10^{-19}\,$ C.
Traduz-se este facto dizendo-se que a carga eléctrica é \emph{quantizada}.

Dentro das várias experiências elaboradas para mostrar este facto, uma montagem clássica é a do físico americano Robert A. Millikan\footnote{Millikan recebeu o prémio Nobel da Física em 1923 pelos seus trabalhos sobre a determinação da carga do electrão e efeito fotoeléctrico.} (1869-1953), também chamada experiência da gota de óleo.
    
\end{center}
\hline


\end{minipage}
\begin{minipage}[c][2cm][c]{\textwidth}
\centering

\end{minipage}

\end{center} 
\cleardoublepage
\fontfamily{cmr}\selectfont
\setcounter{page}{0}
%TC:endignore
% #############################################################################
%
%                           BEGIN MAIN DOCUMENT BODY
%
% #############################################################################

% -----------------------------------------------------------------------------
%                   Primeira Parte 
% -----------------------------------------------------------------------------
\section{Report}
Nesta atividade experimental, procedeu-se à substituição das lentes e da calha utilizada para montar a experiência,
devido às instabilidades observadas com a calha anterior. Optou-se por uma calha de precisão em alumínio triangular, utilizando
os suportes correspondentes para assegurar uma montagem sólida.

A introdução dessa nova calha, aliada aos suportes apropriados, resultou numa significativa melhoria na estabilidade do sistema ótico,
eliminando oscilações notáveis durante a execução da experiência. Vale destacar que, com esta calha, as distâncias marcadas
são representadas por números inteiros.

Quanto às lentes, foram substituídas por outras cujos suportes possibilitam um alinhamento mais fácil, permitindo ajustes
em três graus de liberdade: eixo X, Y e Z. Anteriormente, a alteração estava limitava à altura e à inclinação da lente,
tornando o alinhamento mais desafiador. A expetativa, com esta alteração, é que, mesmo diante de desalinhamentos durante a atividade,
os alunos tenham maior facilidade em corrigi-los.

Para além destes aspetos, o guia desta atividade experimental não foi alterado, sendo que apenas o material foi
substituído. A teoria e o procedimento experimental permanecem os mesmos, assim como as questões colocadas no final do guia.
A única alteração foi a substituição das imagens do guia que mostravam a experiência com a calha anterior, por imagens
da experiência com a nova calha. Contudo, estas imagens deverão ser posteriormente alteradas aquando da decisão do 
uso ou não deste material.

\section{Imagens}

\begin{figure}[H]
    \centering
    \includegraphics[width=0.6\textwidth]{IMG_2862.jpg}
    \caption{Montagem da experiência com a nova calha e as novas lentes.}
    \label{fig:montagem}
\end{figure}

\begin{figure}[H]
    \centering
    \includegraphics[width=0.6\textwidth]{IMG_2863.jpg}
    \caption{Montagem da experiência com a nova calha e as novas lentes.}
    \label{fig:montagem}
\end{figure}

\begin{figure}[H]
    \centering
    \includegraphics[width=0.6\textwidth]{IMG_2864.jpg}
    \caption{Montagem da experiência com a nova calha e as novas lentes.}
    \label{fig:montagem}
\end{figure}

\begin{figure}[H]
    \centering
    \includegraphics[width=0.6\textwidth]{IMG_2868.jpg}
    \caption{Montagem da experiência com a nova calha e as novas lentes.}
    \label{fig:montagem}
\end{figure}

\begin{figure}[H]
    \centering
    \includegraphics[width=0.6\textwidth]{IMG_2871.jpg}
    \caption{Montagem da experiência com a nova calha e as novas lentes.}
    \label{fig:montagem}
\end{figure}

\begin{figure}[H]
    \centering
    \includegraphics[width=0.6\textwidth]{IMG_2875.jpg}
    \caption{Montagem da experiência com a nova calha e as novas lentes.}
    \label{fig:montagem}
\end{figure}


\newpage

\end{document}


% ########################################################################################################################################
%
% This is the main latex file. Here we call for inputs from other files. We also define some of the main characteristics of the document.
%
% ########################################################################################################################################
%
% You likely only need to modify the "Main_Content__Write_your_essay_here.tex" file.
%
% ########################################################################################################################################

\documentclass[12pt,a4paper,oneside]{paper}
% Encoding and Language
\usepackage[utf8x]{inputenc}
\usepackage{csquotes}
\usepackage[main=english,portuguese]{babel}
\usepackage{iflang}
\usepackage{ragged2e}
\usepackage{makeidx}
\usepackage{multicol}
\usepackage[export]{adjustbox}
\usepackage{tikz-dependency}
\usepackage{fancyhdr}

% Font Configurations
\renewcommand{\rmdefault}{phv}
\renewcommand{\sfdefault}{phv}
\def\FontLn{% 16 pt normal
  \usefont{T1}{phv}{m}{n}\fontsize{16pt}{16pt}\selectfont}
\def\FontLb{% 16 pt bold
  \usefont{T1}{phv}{b}{n}\fontsize{16pt}{16pt}\selectfont}
\def\FontMn{% 14 pt normal
  \usefont{T1}{phv}{m}{n}\fontsize{14pt}{14pt}\selectfont}
\def\FontMb{% 14 pt bold
  \usefont{T1}{phv}{b}{n}\fontsize{14pt}{14pt}\selectfont}
\def\FontSn{% 12 pt normal
  \usefont{T1}{phv}{m}{n}\fontsize{12pt}{12pt}\selectfont}

% Font Encoding
\usepackage[T1]{fontenc}

% Page Geometry
\usepackage{geometry}	
\geometry{verbose,tmargin=2.cm,bmargin=2.cm,lmargin=2.cm,rmargin=2cm}

% Line Spacing
\usepackage{setspace}
\renewcommand{\baselinestretch}{1.25}

% Graphics and Figures
\usepackage{graphicx}
\usepackage{subfigure}
\usepackage{subfigmat}
\usepackage{float}

% Mathematics and Theorems
\usepackage{amsmath}
\usepackage{amsthm}
\usepackage{amsfonts}
\usepackage{dcolumn}
\usepackage{indentfirst}

% Comments and Verbatim
\usepackage{verbatim}

% Hyperlinks
\usepackage[pdftex]{hyperref}
\hypersetup{
    colorlinks,
    linkcolor=blue,
    anchorcolor=black,
    citecolor=cyan,
    filecolor=black,
    menucolor=black,
    urlcolor=teal,
    bookmarksopen=true,
    bookmarksnumbered=true
}

% Captions and References
\usepackage[figure,table]{hypcap}
\usepackage[format=plain]{caption}
\DeclareCaptionFont{georgia}{\small\fontseries{n}\fontfamily{georgia}\selectfont}
\captionsetup{labelfont=georgia,font=georgia}

% Bibliography
\usepackage[backend=biber,style=apa]{biblatex}

% Acronyms
\usepackage[printonlyused]{acronym}

% Lipsum (for placeholder text)
\usepackage{lipsum}

% Cleveref (for clever references)
\usepackage[\IfLanguageName{english}{english}{portuguese}]{cleveref}

% Colors
\usepackage{xcolor}
\usepackage{color}

% Custom Commands
\newcommand{\gray}[1]{\textcolor{gray}{#1}}

% Equation Numbering
\renewcommand{\theequation}{{\fontseries{n}\fontfamily{georgia}\selectfont\arabic{equation}}}

% Section and Subsection Fonts
\sectionfont{\Large\bfseries\fontfamily{lmss}\selectfont}
\subsectionfont{\large\bfseries\fontfamily{lmss}\selectfont}

\makeindex

\addbibresource{bibliography.bib}
\begin{document}
\pagestyle{plain}

%TC:ignore
% #############################################################################
%
%                           ENTER YOUR NAME, ISTid, AND TITLE
% 
% #############################################################################



\def\title {Millikan}


% #############################################################################
%
%               DO NOT MODIFY THE LINES FROM HERE TO THE MAIN DOCUMENT BODY
% 
% #############################################################################

\thispagestyle {empty}
\begin{center}
\begin{minipage}[c][5cm][t]{\textwidth}
\begin{center}
\includegraphics[width=5cm]{../IST_A_RGB_POS.png}
\end{center}

\end{minipage}
\begin{minipage}[t][10cm][c]{\textwidth}
\centering
{\FontMb Laboratório de Introdução à Física Experimental} \\
\paragraph{}
\centering
{\FontLb\Huge \title{Experiência de Milikan}}
\paragraph{}
{\FontMb Estimativa da carga elétrica de gotículas de óleo eletrizadas em suspensão num fluido} \\
\paragraph{}
{\FontMb 2023}
\end{minipage}

\begin{minipage}[c][1.5cm][c]{\textwidth}
\centering
{\FontLn }
\end{minipage}

\begin{minipage}[c][1.5cm][c]{\textwidth}
\centering



\end{minipage}
\begin{minipage}[c][3cm][c]{\textwidth}
\centering
\renewcommand{\arraystretch}{1.4}

\maketitle

\vspace{-5mm}
\hline
\vspace{-3mm}
\begin{center}
\centering
\section*{\centering Objetivos}
    \vspace{-3mm}
\small
\justify
Pretende-se com este trabalho determinar a carga eléctrica de pequenas gotas de óleo, tendo como objetivo final mostrar que a carga eléctrica não aparece com uma quantidade qualquer mas sempre como um múltiplo de uma unidade fundamental: a carga do electrão. Deste modo, um corpo electrizado apresenta um excesso de carga de sinal positivo ou negativo, mas cuja valor é sempre um múltiplo do valor da carga elementar $q_{ele}= 1,602176634\cdot 10^{-19}\,$ C.
Traduz-se este facto dizendo-se que a carga eléctrica é \emph{quantizada}.

Dentro das várias experiências elaboradas para mostrar este facto, uma montagem clássica é a do físico americano Robert A. Millikan\footnote{Millikan recebeu o prémio Nobel da Física em 1923 pelos seus trabalhos sobre a determinação da carga do electrão e efeito fotoeléctrico.} (1869-1953), também chamada experiência da gota de óleo.
    
\end{center}
\hline


\end{minipage}
\begin{minipage}[c][2cm][c]{\textwidth}
\centering

\end{minipage}

\end{center} 
\cleardoublepage
\fontfamily{cmr}\selectfont
\setcounter{page}{0}
%TC:endignore
% #############################################################################
%
%                           BEGIN MAIN DOCUMENT BODY
%
% #############################################################################

% -----------------------------------------------------------------------------
%                   Primeira Parte 
% -----------------------------------------------------------------------------
\section{Report}
Nesta atividade experimental, o objetivo principal consiste em substituir a abordagem manual por um sistema computacional,
eliminando assim os possíveis erros associados à visualização das gotas e à leitura no microscópio. Além disso,
essa abordagem proporcionaria a vantagem de não exigir ajustes frequentes na objetiva ao trocar de observador,
uma vez que o computador estaria encarregado da leitura.

Ao adotar um sistema de gravação de imagem, teríamos a facilidade de acompanhar o movimento da gota de maneira mais eficaz
e cronometrar com precisão o tempo que ela leva para percorrer uma distância conhecida.

Apesar das diversas tentativas realizadas com diferentes softwares e câmeras, a conexão do sistema de obtenção de imagem
não foi bem-sucedida. No entanto, o Professor Gonçalo Figueira conseguiu viabilizar um sistema funcional para obtenção de imagem.

A recomendação para esta atividade experimental é iniciar a utilização do computador para capturar as imagens, e, portanto,
o guia foi ajustado para destacar essa abordagem, tendo sido modificado o Procedimento Experimental nas partes em que se
faz referência à visualização da gota e à cronometragem. Vale ressaltar que não houve alterações significativas na teoria;
apenas a metodologia de acompanhamento das gotas e cronometragem foi modificada.

Deste modo, as alterações realizadas no guia correspondem apenas a pequenos ajustes no Procedimento Experimental.


\clearpage



\appendix


\newpage

\end{document}


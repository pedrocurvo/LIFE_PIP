\thispagestyle {empty}
\begin{center}
\begin{minipage}[c][5cm][t]{\textwidth}
\begin{center}
\includegraphics[width=5cm]{./otica_images/IST_A_RGB_POS.png}
\end{center}

\end{minipage}
\begin{minipage}[t][10cm][c]{\textwidth}
\centering
{\FontMb Laboratório de Introdução à Física Experimental} \\
\paragraph{}
\centering
{\FontLb\Huge \title{Ótica Geométrica}}
\paragraph{}
{\FontMb Sistema Ótico Introdutório} \\
\paragraph{}
{\FontMb 2023}
\end{minipage}

\begin{minipage}[c][1.5cm][c]{\textwidth}
\centering
{\FontLn }
\end{minipage}

\begin{minipage}[c][1.5cm][c]{\textwidth}
\centering



\end{minipage}
\begin{minipage}[c][3cm][c]{\textwidth}
\centering
\renewcommand{\arraystretch}{1.4}

\maketitle

\vspace{-5mm}
\hline
\vspace{-3mm}
\begin{center}
\centering
\section*{\centering Objetivos}
    \vspace{-3mm}
\small
\justify
Pretende-se estudar vários aspectos da luz do ponto de vista da óptica geométrica, tais como a reflexão e refracção
entre meios, a polarização, lentes delgadas e associações de lentes. Iremos estudar a formação de imagens reais e virtuais,
verificar como estas dependem das distâncias envolvidas no sistema óptico, e testar um microscópio composto.

    
\end{center}
\hline


\end{minipage}
\begin{minipage}[c][2cm][c]{\textwidth}
\centering

\end{minipage}

\end{center}